\section{Conclusiones}

La instalación de Linux, específicamente Pop! OS, en la Asus TUF Gaming A15 fue un proceso bien estructurado y documentado. Se inició con una investigación previa para garantizar la compatibilidad del hardware, destacando que Pop! OS es una opción ideal debido a su optimización para tarjetas gráficas NVIDIA y su enfoque en la productividad.

La preparación del equipo incluyó el respaldo de datos y la descarga de los prerequisitos necesarios, como la imagen ISO del sistema y la herramienta Rufus para la creación de un USB booteable. Posteriormente, se realizó el flasheo de la USB y la instalación del sistema operativo, configurando correctamente las particiones del disco para un rendimiento óptimo.

En conclusión, el procedimiento seguido permitió una instalación exitosa y funcional de Pop! OS, asegurando estabilidad y rendimiento en la laptop. La documentación detallada de cada paso proporciona una guía útil para futuras instalaciones o referencias.
